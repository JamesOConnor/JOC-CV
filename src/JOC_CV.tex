%%%%%%%%%%%%%%%%%%%%%%%%%%%%%%%%%%%%%%%
% Nikhil Ranjan - 2014 Resume
% LaTeX Template
% Version 1.1 (30/4/2014)
%
%%%%%%%%%%%%%%%%%%%%%%%%%%%%%%%%%%%%%%
%
% Credits:
% https://news.ycombinator.com/item?id=3783538
% https://github.com/deedydas/Deedy-Resume
%
%%%%%%%%%%%%%%%%%%%%%%%%%%%%%%%%%%%%%%

\documentclass[]{JOC_CV}


\begin{document}

%%%%%%%%%%%%%%%%%%%%%%%%%%%%%%%%%%%%%%
%
%     LAST UPDATED DATE
%
%%%%%%%%%%%%%%%%%%%%%%%%%%%%%%%%%%%%%%

%%%%%%%%%%%%%%%%%%%%%%%%%%%%%%%%%%%%%%
%
%     TITLE NAME
%
%%%%%%%%%%%%%%%%%%%%%%%%%%%%%%%%%%%%%%


\namesection{James}{O'Connor}{ \urlstyle{same}\url{http://jp-oconnor.com} \\
\href{mailto:k1454695@kingston.ac.uk}{james.oconnor@kingston.ac.uk} | Phone number masked}

%%%%%%%%%%%%%%%%%%%%%%%%%%%%%%%%%%%%%%
%
%     COLUMN ONE
%
%%%%%%%%%%%%%%%%%%%%%%%%%%%%%%%%%%%%%%

\begin{minipage}[t]{0.33\textwidth}

%%%%%%%%%%%%%%%%%%%%%%%%%%%%%%%%%%%%%%
%     EDUCATION
%%%%%%%%%%%%%%%%%%%%%%%%%%%%%%%%%%%%%%

\section{Projects}\label{sec:projects}

\subsection{Sentinel bot}\label{subsec:sentinel-bot}
\location{Twitter account which posts a Sentinel-2 satellite image every 30 minutes |} \href{http://twitter.com/sentinel\_bot}{http://twitter.com/sentinel\_bot}
\subsection{Personal blog}\label{subsec:personal-blog}
\location{Details aspects of my research, with explanations and tutorials |}\href{jamesoconnorkingston.wordpress.com}{jamesoconnorkingston.wordpress.com}
\subsection{Greyscaler}\label{subsec:greyscaler}
\location{Esoteric greyscaling algorithms via django web app |}\href{tinyurl.com/greyscaler}{tinyurl.com/greyscaler}
\sectionsep


%%%%%%%%%%%%%%%%%%%%%%%%%%%%%%%%%%%%%%
%     COURSEWORK
%%%%%%%%%%%%%%%%%%%%%%%%%%%%%%%%%%%%%%

\section{Expertise}\label{sec:expertise}
Python \\
Web Development (Django) \\
GIS/Mapping \\
Docker \\
Kubernetes \\
Predictive Modelling \\
Scikit-learn \\
Tensor Flow \\
OpenCV \\
Image Processing \\
Data Science \\
Databases \\
\sectionsep

%%%%%%%%%%%%%%%%%%%%%%%%%%%%%%%%%%%%%%
%     LINKS
%%%%%%%%%%%%%%%%%%%%%%%%%%%%%%%%%%%%%%

\section{Links}\label{sec:links}
\href{https://www.linkedin.com/in/jamesoconnor12/}{LinkedIn://jamesoconnor12} \\
\href{https://github.com/JamesOConnor}{Github://JamesOConnor} \\
\href{https://twitter.com/James\_o\_connor1}{Twitter://\@James\_O\_Connor1} \\
\href{https://jamesoconnorkingston.wordpress.com}{Wordpress://jamesoconnorkingston} \\
\href{http://www.jp-oconnor.com}{Website://jp-oconnor.com} \\
\href{https://www.kaggle.com/james0c}{Kaggle://james0c}
\sectionsep

%%%%%%%%%%%%%%%%%%%%%%%%%%%%%%%%%%%%%%
%     SKILLS
%%%%%%%%%%%%%%%%%%%%%%%%%%%%%%%%%%%%%%

\section{Skills}\label{sec:skills}
\subsection{Computing}\label{subsec:computing}
\location{Languages}
Python \textbullet{} Linux \textbullet{} Django \textbullet{} Javascript \textbullet{} Imagemagick \textbullet{} MSSQL \textbullet{} git \linebreak
\location{Software packages}
Microsoft office \textbullet{} PyCharm IDE \textbullet{} ArcGIS/QGIS
\sectionsep

%%%%%%%%%%%%%%%%%%%%%%%%%%%%%%%%%%%%%%
%     Certification
%%%%%%%%%%%%%%%%%%%%%%%%%%%%%%%%%%%%%%

\section{Languages}\label{sec:languages}
English (Native) \\
Spanish (Fluent) \\
French (Basic) \\
\sectionsep

%%%%%%%%%%%%%%%%%%%%%%%%%%%%%%%%%%%%%%
%
%     COLUMN TWO
%
%%%%%%%%%%%%%%%%%%%%%%%%%%%%%%%%%%%%%%

\end{minipage}
\hfill
\begin{minipage}[t]{0.66\textwidth}

%%%%%%%%%%%%%%%%%%%%%%%%%%%%%%%%%%%%%%
%     EXPERIENCE
%%%%%%%%%%%%%%%%%%%%%%%%%%%%%%%%%%%%%%

\section{Education}\label{sec:education}

\runsubsection{Kingston University London}
\descript{| PhD Candidate }
\location{September 2014 - Present | London, UK}
% Hacky fix for awkward extra vertical space
Thesis entitled 'Optimisation of aerial surveying from Unmanned Aerial Vehicles (UAVs)'.
Used custom image processing techniques developed using python OpenCV and photogrammetric packages for generation of detailed 3D data using Structure-from-Motion (SfM) photogrammetry.
\sectionsep

\runsubsection{University College London}
\descript{| Remote Sensing Masters | Distinction}
\location{September 2013 - September 2014 | London, UK}
Thesis investigated the accuracy of a radiative transfer model (Joint Research Council's two-stream inversion package).
Used large amounts of satellite data (MODIS) for comparison using custom python scripts.
\sectionsep

\runsubsection{Trinity College Dublin}
\descript{| Bachelor of Science | 2:1}
\location{September 2006 - September 2010 | Dublin, Ireland}
Thesis investigated biodiversity of urban pollinators in Dublin city centre and methods for promoting diversity.
Fieldwork skills in generating data for Environmental Impact Assessments.
\sectionsep

%%%%%%%%%%%%%%%%%%%%%%%%%%%%%%%%%%%%%%
%     Projects
%%%%%%%%%%%%%%%%%%%%%%%%%%%%%%%%%%%%%%

\section{Experience}\label{sec:experience}

\runsubsection{Earth Observation Data Scientist - KisanHub}
\linebreak
\location{February 2018 - present}
Interacted with multiple external APIs for provisioning of satellite imagery from different sources for KisanHub's web application.
Automated image processing pipelines to deliver processed data via a microservice.
Trained and tested models to provide more accurate feedback on land-surface state.
Followed AGILE methodologies for software development.
\sectionsep

\runsubsection{RSPSOC SENSED Editor}
\linebreak
\location{June 2015 - present}
Have sat in council meetings and engaged in discussions on how to grow the society.
Organised and ran a 3 day conference at Kingston in March 2017 as part of my duties.
SENSED is a quarterly magazine produced by the society, of which I am an editor.
\sectionsep

\runsubsection{Developer - Quartz syndicate}
\linebreak
\location{November 2015 - November 2017}
Used python scikit-learn toolbox to train models in prediction of outcomes of sporting events.
Specific tools include logistic regression, tree-based learners (LightGBM) and tensor-flow neural
networks to refine predictions.
\sectionsep


%%%%%%%%%%%%%%%%%%%%%%%%%%%%%%%%%%%%%%
%     AWARDS
%%%%%%%%%%%%%%%%%%%%%%%%%%%%%%%%%%%%%%
\section{Papers}\label{sec:papers}

\begin{tabular}{rll}
2017	& Cameras and settings for aerial surveys in the geosciences: optimizing image data \\
&Progress in Physical Geography - \href{https://dx.doi.org/10.1177/0309133317703092}{Link to paper}
 \\
\end{tabular}
\sectionsep

%%%%%%%%%%%%%%%%%%%%%%%%%%%%%%%%%%%%%%
%     SOCIETIES
%%%%%%%%%%%%%%%%%%%%%%%%%%%%%%%%%%%%%%

\section{Societies}\label{sec:societies}

\begin{tabular}{rll}
2014 - Present	& Remote Sensing and Photogrammetry Society\\
2016 - Present	& European Geophysical Union\\
\end{tabular}
\sectionsep

\end{minipage}
\end{document}  \documentclass[]{article}